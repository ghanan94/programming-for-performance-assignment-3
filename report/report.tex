\documentclass[12pt]{article}

\usepackage[letterpaper, hmargin=0.75in, vmargin=0.75in]{geometry}
\usepackage{float}
\usepackage{listings}

\pagestyle{empty}

\title{ECE 459: Programming for Performance\\Assignment 3}
\author{Ghanan Gowripalan}
\date{\today}

% Code listing style
\lstset{frame=single}

\begin{document}

\maketitle

\section*{Part 1: Brute-force Approach}

\begin{table}[H]
  \centering
  \begin{tabular}{lr}
    & {\bf Time (s)} \\
    \hline
    Run 1 & 10.787 \\
    Run 2 & 10.836 \\
    Run 3 & 10.790 \\
    \hline
    Average & 10.804
  \end{tabular}
  \caption{Benchmark results for sequential N-Body implementation with 500 points and no approximations}
  \label{tbl-nbody-seq-500-no-approx}
\end{table}

\begin{table}[H]
  \centering
  \begin{tabular}{lr}
    & {\bf Time (s)} \\
    \hline
    Run 1 & 0.211 \\
    Run 2 & 0.195 \\
    Run 3 & 0.175 \\
    \hline
    Average & 0.194
  \end{tabular}
  \caption{Benchmark results for OpenCl N-Body implementation with 500 points and no approximations}
  \label{tbl-nbody-gpu-500-no-approx}
\end{table}

\begin{table}[H]
  \centering
  \begin{tabular}{lr}
    & {\bf Time (s)} \\
    \hline
    Run 1 & 0.0 \\
    Run 2 & 0.0 \\
    Run 3 & 0.0 \\
    \hline
    Average & 0.0
  \end{tabular}
  \caption{Benchmark results for sequential N-Body implementation with 5000 points and no approximations}
  \label{tbl-nbody-seq-5000-no-approx}
\end{table}

\begin{table}[H]
  \centering
  \begin{tabular}{lr}
    & {\bf Time (s)} \\
    \hline
    Run 1 & 4.524 \\
    Run 2 & 4.507 \\
    Run 3 & 4.502 \\
    \hline
    Average & 4.511
  \end{tabular}
  \caption{Benchmark results for OpenCl N-Body implementation with 5000 points and no approximations}
  \label{tbl-nbody-gpu-5000-no-approx}
\end{table}


\section*{Part 2: Far-field approximations}

\begin{table}[H]
  \centering
  \begin{tabular}{lr}
    & {\bf Time (s)} \\
    \hline
    Run 1 & 2.015 \\
    Run 2 & 2.015 \\
    Run 3 & 2.041 \\
    \hline
    Average & 2.024
  \end{tabular}
  \caption{Benchmark results for sequential N-Body implementation with 500 points with approximations}
  \label{tbl-nbody-seq-500-approx}
\end{table}

\begin{table}[H]
  \centering
  \begin{tabular}{lr}
    & {\bf Time (s)} \\
    \hline
    Run 1 & 0.178 \\
    Run 2 & 0.160 \\
    Run 3 & 0.159 \\
    \hline
    Average & 0.194
  \end{tabular}
  \caption{Benchmark results for OpenCl N-Body implementation with 500 points with approximations}
  \label{tbl-nbody-gpu-500-approx}
\end{table}

\begin{table}[H]
  \centering
  \begin{tabular}{lr}
    & {\bf Time (s)} \\
    \hline
    Run 1 & 84.105 \\
    Run 2 & 84.629 \\
    Run 3 & 84.967 \\
    \hline
    Average & 84.567
  \end{tabular}
  \caption{Benchmark results for sequential N-Body implementation with 5000 points with approximations}
  \label{tbl-nbody-seq-5000-approx}
\end{table}

\begin{table}[H]
  \centering
  \begin{tabular}{lr}
    & {\bf Time (s)} \\
    \hline
    Run 1 & 5.368 \\
    Run 2 & 5.383 \\
    Run 3 & 5.352 \\
    \hline
    Average & 5.368
  \end{tabular}
  \caption{Benchmark results for OpenCl N-Body implementation with 5000 points with approximations}
  \label{tbl-nbody-gpu-5000-approx}
\end{table}



\begin{lstlisting}
#include <iostream>

int main(int argc, char *argv[])
{
    std::cout << "Example code listing" << std::endl;
}
\end{lstlisting}

\end{document}
